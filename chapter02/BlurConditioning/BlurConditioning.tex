\documentclass[11pt]{article}

\usepackage{amsmath,amsthm}
\usepackage[headings]{fullpage}
\usepackage[utopia]{mathdesign}
\usepackage{color}

\pagestyle{myheadings}
\markboth{Matrix conditioning}{}

\usepackage{amsmath}
\usepackage{bm}
\usepackage[T1]{fontenc}
\usepackage{matlab-prettifier}

\newcommand{\nat}{\mathbb{N}}          % Natural numbers
\newcommand{\integer}{\mathbb{Z}}      % Integers
\newcommand{\real}{ {\mathbb{R}} }     % Reals
\newcommand{\float}{ {\mathbb{F}} }     % Reals
\newcommand{\rmn}[2]{ \mathbb{R}^{#1\times#2} }     % Reals
\newcommand{\complex}{ {\mathbb{C}} }  % Complex
\newcommand{\macheps}{\ensuremath \varepsilon_{\text{mach}}}

\renewcommand{\Re}{\operatorname{Re}}
\renewcommand{\Im}{\operatorname{Im}}

% Boldface vectors
\newcommand{\bff}{\bm{f}}
\newcommand{\bfF}{\bm{F}}
\newcommand{\bfw}{\bm{w}}
\newcommand{\bfv}{\bm{v}}
\newcommand{\bfe}{\bm{e}}
\newcommand{\bfc}{\bm{c}}
\newcommand{\bfp}{\bm{p}}
\newcommand{\bfq}{\bm{q}}
\newcommand{\bfr}{\bm{r}}
\newcommand{\bfs}{\bm{s}}
\newcommand{\bfu}{\bm{u}}
\newcommand{\bfb}{\bm{b}}
\newcommand{\bfx}{\bm{x}}
\newcommand{\bfy}{\bm{y}}
\newcommand{\bfg}{\bm{g}}
\newcommand{\bfh}{\bm{h}}
\newcommand{\bfz}{\bm{z}}
\newcommand{\bfa}{\bm{a}}
\newcommand{\bft}{\bm{t}}
\newcommand{\bfd}{\bm{d}}
\newcommand{\bfalpha}{\bm{\alpha}}
\newcommand{\bfeps}{\bm{\varepsilon}}
\newcommand{\bfdelta}{\bm{\delta}}
\newcommand{\bfzero}{\bm{0}}
\newcommand{\eye}[1]{\bfe_{#1}}

% Boldface matrix
\newcommand{\m}[1]{\bm{#1}}
\newcommand{\mA}{\m{A}}
\newcommand{\mL}{\m{L}}
\newcommand{\mF}{\m{F}}
\newcommand{\mU}{\m{U}}
\newcommand{\mJ}{\m{J}}
\newcommand{\mP}{\m{P}}
\newcommand{\mQ}{\m{Q}}
\newcommand{\mR}{\m{R}}
\newcommand{\mD}{\m{D}}
\newcommand{\mS}{\m{S}}
\newcommand{\mB}{\m{B}}
\newcommand{\mC}{\m{C}}
\newcommand{\mE}{\m{E}}
\newcommand{\mG}{\m{G}}
\newcommand{\mH}{\m{H}}
\newcommand{\mV}{\m{V}}
\newcommand{\mW}{\m{W}}
\newcommand{\mX}{\m{X}}
\newcommand{\mZ}{\m{Z}}
\newcommand{\mK}{\m{K}}
\newcommand{\mM}{\m{M}}

\newcommand{\meye}{\m{I}}

\newcommand{\ee}[1]{\times 10^{#1}}
\newcommand{\jac}[2]{\frac{\bfd \bm{#1}}{\bfd \bm{#2}}}
\newcommand{\diag}{\operatorname{diag}}
\newcommand{\fl}{\operatorname{fl}}
\newcommand{\circop}[1]{\makebox[0pt][l]{$\bigcirc$}\hspace{1pt}#1}
\newcommand{\myvec}{\operatorname{vec}}
\newcommand{\unvec}{\operatorname{unvec}}
\newcommand{\kron}[2]{#1 \otimes #2}

% matlab stuff
\lstset{style=Matlab-editor,basicstyle=\mlttfamily}
\newcommand{\Mfile}[1]{\lstinputlisting[]{#1}}
\lstnewenvironment{matlab}%
  {\lstset{escapechar=`}}%
  {}
\lstMakeShortInline[style=Matlab-editor,basicstyle=\mlttfamily]"  % use " for inline code
\newcommand{\matlabend}{\lstinline[style=Matlab-style,basicstyle=\mlttfamily,mloverride=true]!end!}


\begin{document}

\begin{center}
  \bf Terms and conditions
\end{center}

Previously you have seen how to blur an image, represented as an
$m\times n$ pixel intensity matrix $\mX$, through multiplying by a matrix on each side:
\begin{equation}
  \label{eq:2}
   \mZ=\mV\, \mX \, \mH,
\end{equation}
where $\mV = (\mB_m)^k$, $\mH = (\mB_n)^k$, 
$B_i$ is the $i\times i$ matrix made by your function
\texttt{blurmatrix}, and $k$ is a positive integer. You have also seen that deblurring can be accomplished by multiplying by matrix inverses on each side of $\mZ$ (as computed equivalently by solving linear systems). However, while the restoration seems perfect for $k=1$, it fails completely for larger $k$.

Matrix condition numbers explain these observations. The blurred matrix $\mZ$ is perturbed by an amount comparable to machine precision. For $k=1$, the condition numbers of $\mV$ and $\mH$ are not large enough to amplify this error up to the same order of magnitude as $\mX$ itself, so the noise is not perceived. But at some $k>1$, the condition numbers are so large that the noise is amplified enough to overwhelm the expected result.

Let $\bfx$ be a single column of $\mX$. For the case of vertical blurring only, we have $\bfz=\mV\bfx$ as a column of $\mZ$. We solve $\mV \bfy = \bfz$ for $\bfy$, which is mathematically the same as $\bfx$. Due to machine precision, though, the perturbation to $\bfz$ causes an error satisfying
\begin{equation}
	\label{bound}
	\frac{\|\bfy-\bfx\|}{\|\bfx\|} \le \kappa(\mV)\macheps,
\end{equation}
with $\kappa$ being the matrix condition number. 

\subsection*{Preparation}

Read Section~2.8. Make sure your working \texttt{blurmatrix.m} is available. 

\subsection*{Goals}

You will compute condition numbers of blur matrix powers and compare them to the errors of repeated blur/deblur operations. 

\subsection*{Procedure}

\begin{enumerate}
\item The condition number of a matrix has two factors, $\|\mA\|$ and $\|\mA^{-1}\|$. First you will show that the $\|\mA\|$ term makes no trouble. For $n=50,100,150,\ldots,800$, plot $\|\mB_n\|$ versus $n$. 

\item For the same $n$ as in step 1, plot $\kappa(\mB_n)$ versus $n$. You should use a log-log scale for this graph and get essentially a straight line. This implies that $\log \kappa \approx a \log n + b$, or $\kappa \approx C n^p$. 

\item Let $\mV=\mB_{100}$. For $k=1,2,\ldots,8$, plot $\kappa(\mV^k)$ as a function of $k$. This time the graph is straight on a semi-log scale, which implies $\kappa \approx C q^k$. 

\item Let $\bfx$ be a random vector of length 100. For $k=1,2,\ldots,8$, let $\bfz=\mV\bfx$ and then solve $\mV\bfy=\bfz$ for $\bfy$. Record the relative error in the result. Then make a table showing both sides of the inequality~\eqref{bound}.

\end{enumerate}

\end{document}

%%% Local Variables:
%%% mode: latex
%%% TeX-master: t
%%% End:
