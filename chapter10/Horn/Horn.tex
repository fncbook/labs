\documentclass[11pt,twoside]{article}

\usepackage{amsmath,amsthm}
\usepackage[headings]{fullpage}
\usepackage[utopia]{mathdesign}
\usepackage{color}
\usepackage{graphicx}
\usepackage[colorlinks=true]{hyperref}

\pagestyle{myheadings}
\markboth{Horn equation}{Horn equation}


\begin{document}
   
\input{../macros}

\begin{center}
    \bf Horn of plenty
\end{center}

The use of mathematics for the design and analysis of musical instruments has a long and rich history. One of the landmarks in this area is the 1919 model known as Webster's horn equation, even though the relevant modeling and mathematics had largely been worked out by such luminaries as Daniel Bernoulli, d'Alembert, Euler, Lagrange, Green, Helmholtz, and Rayleigh. 

During a sustained note, the variation of the air pressure in a horn quickly settles into a \textbf{standing wave}, in which a fixed spatial pattern oscillates in time at a harmonic frequency. (You can see \href{https://youtu.be/ShLDhqM0gR4?si=4enCHV6OBk9-mTs4&t=39}{standing waves on a guitar string} or \href{https://youtu.be/e6JLXZN3Ov0?si=5XwtqKBd3p0g_lzn&t=20}{in a pool}.) The frequency of the vibration determines the pitch of the resulting sound.

In Webster's model, the unknown air pressure $u(x)$ in a long, thin tube of length $L$ is governed by
\begin{equation}
  \label{eq:webster}
  u'' + \frac{A'}{A} u' + \omega^2 u = 0, \qquad x\in [0,L],
\end{equation}
where $\omega$ is the harmonic frequency and $A(x)$ is the cross-sectional area of the tube. Reasonable boundary conditions are a fixed nonzero value of $u$ at $x=0$ (the mouthpiece) and $u'(L)=0$ at the open end. In the analysis problem, $A(x)$ is prescribed, whereas in a design problem the goal is to find an $A(x)$ that optimizes some desired quality of the horn. 

For a few prescriptions of $A(x)$, a solution can be found in closed form. For example, if $A(x)=e^{2k x}$, then the general solution of~(\ref{eq:webster}) is
\begin{equation}
  \label{eq:solution-exp}
  u(x) = c_1 e^{s_1 x} + c_2 e^{s_2 x},
\end{equation}
where $s_1$ and $s_2$ are the roots of 
\begin{equation}
  \label{eq:roots}
  s^2 + 2k s + \omega^2 = 0, 
\end{equation}
and $c_1$, $c_2$ are constants determined by imposing the boundary conditions:
\begin{equation}
  \label{eq:bc}
  \begin{bmatrix}
    1 & 1 \\ s_1 e^{s_1 L} & s_2 e^{s_2 L}
  \end{bmatrix}
  \begin{bmatrix}
    c_1 \\ c_2
  \end{bmatrix}
  =
  \begin{bmatrix}
    1 \\ 0
  \end{bmatrix}.
\end{equation}

In most cases, only a numerical solution is feasible. The problem is a linear BVP and easy to solve for most choices of $A(x)$. The solution amplitude 
\begin{equation}
  \label{eq:amplitude}
  \alpha(\omega)=\|u\|_\infty  
\end{equation}
can vary considerably with the frequency $\omega$, and a plot of $\alpha(\omega)$ often shows distinct sharp peaks at \textbf{resonant} frequencies. Physically, a horn player produces white noise by buzzing into the mouthpiece. This noise has significant energy across a wide range of frequencies, but only the resonant frequencies are amplified to make a musical note with a pitch and overtones. 

\subsection*{Goals}

You will compute finite difference solutions of the horn equation for a few choices of $A(x)$ and look for resonances. Although the BVP is linear, you will have to use the general nonlinear solver \texttt{bvp()} from the textbook, because the linear solver is not set up to handle a boundary condition on the derivative of the solution.

\subsection*{Procedure}

\begin{enumerate}
    \item Set $L=2$, $\omega=5$, and $A(x)=e^{6x}$  in~(\ref{eq:webster}). Find the roots $s_1$ and $s_2$ in~(\ref{eq:solution-exp}). Use \eqref{eq:bc} to compute the constants $c_1$ and $c_2$, and plot the exact solution for $0\le x \le L$. (The roots are complex numbers. That's fine, and it produces a mathematically real solution, but you will have to extract the real part of $u$ explicitly due to roundoff in the imaginary part.) 
    \item Using the same values as in step~1, apply \texttt{bvp} from the textbook with $n=240$ to find a numerical solution. Plot the difference between the exact and finite-difference solutions at the nodes. 
    \item Now let $A(x)=3x^2+x+1$ and $\omega=2$. Plot the finite difference solution using $n=240$. (Check visually that the boundary conditions appear to be satisfied.)
    \item Repeat the computation from step~3 but for 140 equally spaced values of $\omega$ between 0.2 and 3. (This should only take a few minutes to complete.) In each case, compute and store $\alpha(\omega)$ using~\eqref{eq:amplitude}. Plot $\alpha$ as a function of $\omega$.
    \item Your amplitude graph should have two sharp peaks. Find the value $\tilde{\omega}$ that maximizes the amplitude, compute the solution for that value, and plot this resonant solution as a function of $x$. It should have one interior local extremum and an amplitude equal to $\alpha(\tilde{\omega})$.
\end{enumerate}

% \subsection*{Extras}
% \begin{enumerate}
%   \item[E1.] Here is a creative way to check the solution even when the exact solution is unknown. Equation~(\ref{eq:webster}) can be rearranged to give
%     \begin{equation}
%       \label{eq:logAprime}
%       \log A(x) = \log A(0) - \int_0^x \frac{u''(s) + \omega^2u(s)}{u'(s)}\, ds.
%     \end{equation}
%     Use $\omega_1$ and $\bfu_1$ from the last step. Obtain the differentiation matrices and compute the vector
% \begin{verbatim}
% z = -(Dxx*u1 + omega1^2*u1) ./ (Dx*u1);
% \end{verbatim}
%     which gives the integrand of~(\ref{eq:logAprime}) at the equally spaced nodes. (The last value, $z_n$, will be useless because of the boundary condition, but this does not affect much.) For each $x_i$, perform trapezoid quadrature on $\bfz$ from element 0 to $i$ in order to approximate the integration in~(\ref{eq:logAprime}). Then finish the calculation of $A(x)$ and compare it graphically to the original $A(x)$ used to define the ODE. 
% \end{enumerate}

\end{document}

