\documentclass[11pt,twoside]{article}

\usepackage{amsmath,amsthm}
\usepackage[headings]{fullpage}
\usepackage[utopia]{mathdesign}
\usepackage{color}
\usepackage{graphicx}

\pagestyle{myheadings}
\markboth{Eye See You}{Eye See You}

\usepackage{amsmath}
\usepackage{bm}
\usepackage[T1]{fontenc}
\usepackage{matlab-prettifier}

\newcommand{\nat}{\mathbb{N}}          % Natural numbers
\newcommand{\integer}{\mathbb{Z}}      % Integers
\newcommand{\real}{ {\mathbb{R}} }     % Reals
\newcommand{\float}{ {\mathbb{F}} }     % Reals
\newcommand{\rmn}[2]{ \mathbb{R}^{#1\times#2} }     % Reals
\newcommand{\complex}{ {\mathbb{C}} }  % Complex
\newcommand{\macheps}{\ensuremath \varepsilon_{\text{mach}}}

\renewcommand{\Re}{\operatorname{Re}}
\renewcommand{\Im}{\operatorname{Im}}

% Boldface vectors
\newcommand{\bff}{\bm{f}}
\newcommand{\bfF}{\bm{F}}
\newcommand{\bfw}{\bm{w}}
\newcommand{\bfv}{\bm{v}}
\newcommand{\bfe}{\bm{e}}
\newcommand{\bfc}{\bm{c}}
\newcommand{\bfp}{\bm{p}}
\newcommand{\bfq}{\bm{q}}
\newcommand{\bfr}{\bm{r}}
\newcommand{\bfs}{\bm{s}}
\newcommand{\bfu}{\bm{u}}
\newcommand{\bfb}{\bm{b}}
\newcommand{\bfx}{\bm{x}}
\newcommand{\bfy}{\bm{y}}
\newcommand{\bfg}{\bm{g}}
\newcommand{\bfh}{\bm{h}}
\newcommand{\bfz}{\bm{z}}
\newcommand{\bfa}{\bm{a}}
\newcommand{\bft}{\bm{t}}
\newcommand{\bfd}{\bm{d}}
\newcommand{\bfalpha}{\bm{\alpha}}
\newcommand{\bfeps}{\bm{\varepsilon}}
\newcommand{\bfdelta}{\bm{\delta}}
\newcommand{\bfzero}{\bm{0}}
\newcommand{\eye}[1]{\bfe_{#1}}

% Boldface matrix
\newcommand{\m}[1]{\bm{#1}}
\newcommand{\mA}{\m{A}}
\newcommand{\mL}{\m{L}}
\newcommand{\mF}{\m{F}}
\newcommand{\mU}{\m{U}}
\newcommand{\mJ}{\m{J}}
\newcommand{\mP}{\m{P}}
\newcommand{\mQ}{\m{Q}}
\newcommand{\mR}{\m{R}}
\newcommand{\mD}{\m{D}}
\newcommand{\mS}{\m{S}}
\newcommand{\mB}{\m{B}}
\newcommand{\mC}{\m{C}}
\newcommand{\mE}{\m{E}}
\newcommand{\mG}{\m{G}}
\newcommand{\mH}{\m{H}}
\newcommand{\mV}{\m{V}}
\newcommand{\mW}{\m{W}}
\newcommand{\mX}{\m{X}}
\newcommand{\mZ}{\m{Z}}
\newcommand{\mK}{\m{K}}
\newcommand{\mM}{\m{M}}

\newcommand{\meye}{\m{I}}

\newcommand{\ee}[1]{\times 10^{#1}}
\newcommand{\jac}[2]{\frac{\bfd \bm{#1}}{\bfd \bm{#2}}}
\newcommand{\diag}{\operatorname{diag}}
\newcommand{\fl}{\operatorname{fl}}
\newcommand{\circop}[1]{\makebox[0pt][l]{$\bigcirc$}\hspace{1pt}#1}
\newcommand{\myvec}{\operatorname{vec}}
\newcommand{\unvec}{\operatorname{unvec}}
\newcommand{\kron}[2]{#1 \otimes #2}

% matlab stuff
\lstset{style=Matlab-editor,basicstyle=\mlttfamily}
\newcommand{\Mfile}[1]{\lstinputlisting[]{#1}}
\lstnewenvironment{matlab}%
  {\lstset{escapechar=`}}%
  {}
\lstMakeShortInline[style=Matlab-editor,basicstyle=\mlttfamily]"  % use " for inline code
\newcommand{\matlabend}{\lstinline[style=Matlab-style,basicstyle=\mlttfamily,mloverride=true]!end!}


\begin{document}

\begin{center}
  \bf Eye See You
\end{center}

We have used least-squares fitting to create functions of a single variable, $y=f(t)$, which can be plotted as a curve to represent data. Some curves, however, cannot be represented as a single function. A more flexible representation is a parametric curve: 
\begin{equation}
  \label{eq:1}
  x = f(t), \qquad y = g(t).
\end{equation}
Given points in the plane as $(x_i,y_i)$, we can separately fit them as functions of a third parametric variable $t$ and use the curve $(f(t),g(t))$ to pass near the points. 

\subsection*{Preparation}

Read section 3.1. 

\subsection*{Goals}

You will capture an image of an eye and find points along the top and bottom eyelids, then do two least-squares fits to represent the eyelids as curves. Because both $x$ and $y$ are periodic as you go around the eye once, you will use periodic functions for the least-squares fitting implied in equation~(1):
\begin{align}
  f(t) &= b_1 + b_2 \cos(2\pi t) + b_3 \cos(4\pi t) + b_4 \cos(6\pi t)  + b_5 \sin(2\pi t) + b_6 \sin(4\pi t) + b_7 \sin(6\pi t),\label{eq:2x} \\
  g(t) &= c_1 + c_2 \cos(2\pi t) + c_3 \cos(4\pi t) + c_4 \cos(6\pi t)  + c_5 \sin(2\pi t) + c_6 \sin(4\pi t) + c_7 \sin(6\pi t). \label{eq:2y}
\end{align}


\subsection*{Procedure}

Download the template script and edit it to perform the following steps.

\begin{enumerate}
\item Using a phone, take a picture of an open eye (your own or someone else's). Load the image into MATLAB using \texttt{imread} and display it using \texttt{image}. 
\item Enter the command 
\begin{verbatim}
[xup,yup] = ginput(10);
\end{verbatim}
This will create a crosshair in the image window. Click at ten roughly
evenly spaced points along the upper eyelid \textbf{from right to left}. Get close to the corners of the eye, but don't put points on the corners. Afterward both \texttt{xup} and \texttt{yup} will be $10\times 1$ vectors representing the selected points. 
\item Repeat step 2 using \verb+[xlo,ylo] = ginput(10)+ and clicking
  along the lower eyelid \textbf{from left to right}.
\item Stack \texttt{xup} and \texttt{xlo} into a vector \texttt{x},
  and stack \texttt{yup} and \texttt{ylo} into a vector \texttt{y}. Both of these should be $20\times 1$. On top of your eye image, plot the points $(x_i,y_i)$ using \texttt{'o'} markers. (If the points don't lie close to the eyelids, you have done something wrong.)  
\item Now let \texttt{t} be a $20\times 1$ vector where $t_i=(i-1)/20$ for $i=1,\ldots,20$. Referring back to equations~\eqref{eq:2x} and~\eqref{eq:2y}, create a $20\times 7$ matrix \texttt{A} whose columns are the values of the functions $1$, $\cos(2\pi t)$, and so on, through $\sin(6\pi t)$. 
\item Apply linear least squares (using backslash) to solve for the coefficients $b_j$ in~(2) using the \texttt{x} data, and to solve for the coefficients $c_j$ in~(3) using the \texttt{y} data. 
\item Evaluate the functions in~\eqref{eq:2x} and~\eqref{eq:2y} at 500 equally spaced values of $t$ between 0 and 1. On top of the axes showing the eye image and the selected points, and using the coefficients from the previous step, plot the curve defined by equation~\eqref{eq:1}. 
\end{enumerate}


\end{document}

%%% Local Variables: 
%%% mode: latex
%%% TeX-master: t
%%% End: 
