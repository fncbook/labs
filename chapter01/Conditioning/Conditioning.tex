\documentclass[11pt]{article}

\usepackage[headings]{fullpage}
\usepackage[utopia]{mathdesign}

\pagestyle{myheadings}
\markboth{}{Conditioning}

\usepackage{amsmath}
\usepackage{bm}
\usepackage[T1]{fontenc}
\usepackage{matlab-prettifier}

\newcommand{\nat}{\mathbb{N}}          % Natural numbers
\newcommand{\integer}{\mathbb{Z}}      % Integers
\newcommand{\real}{ {\mathbb{R}} }     % Reals
\newcommand{\float}{ {\mathbb{F}} }     % Reals
\newcommand{\rmn}[2]{ \mathbb{R}^{#1\times#2} }     % Reals
\newcommand{\complex}{ {\mathbb{C}} }  % Complex
\newcommand{\macheps}{\ensuremath \varepsilon_{\text{mach}}}

\renewcommand{\Re}{\operatorname{Re}}
\renewcommand{\Im}{\operatorname{Im}}

% Boldface vectors
\newcommand{\bff}{\bm{f}}
\newcommand{\bfF}{\bm{F}}
\newcommand{\bfw}{\bm{w}}
\newcommand{\bfv}{\bm{v}}
\newcommand{\bfe}{\bm{e}}
\newcommand{\bfc}{\bm{c}}
\newcommand{\bfp}{\bm{p}}
\newcommand{\bfq}{\bm{q}}
\newcommand{\bfr}{\bm{r}}
\newcommand{\bfs}{\bm{s}}
\newcommand{\bfu}{\bm{u}}
\newcommand{\bfb}{\bm{b}}
\newcommand{\bfx}{\bm{x}}
\newcommand{\bfy}{\bm{y}}
\newcommand{\bfg}{\bm{g}}
\newcommand{\bfh}{\bm{h}}
\newcommand{\bfz}{\bm{z}}
\newcommand{\bfa}{\bm{a}}
\newcommand{\bft}{\bm{t}}
\newcommand{\bfd}{\bm{d}}
\newcommand{\bfalpha}{\bm{\alpha}}
\newcommand{\bfeps}{\bm{\varepsilon}}
\newcommand{\bfdelta}{\bm{\delta}}
\newcommand{\bfzero}{\bm{0}}
\newcommand{\eye}[1]{\bfe_{#1}}

% Boldface matrix
\newcommand{\m}[1]{\bm{#1}}
\newcommand{\mA}{\m{A}}
\newcommand{\mL}{\m{L}}
\newcommand{\mF}{\m{F}}
\newcommand{\mU}{\m{U}}
\newcommand{\mJ}{\m{J}}
\newcommand{\mP}{\m{P}}
\newcommand{\mQ}{\m{Q}}
\newcommand{\mR}{\m{R}}
\newcommand{\mD}{\m{D}}
\newcommand{\mS}{\m{S}}
\newcommand{\mB}{\m{B}}
\newcommand{\mC}{\m{C}}
\newcommand{\mE}{\m{E}}
\newcommand{\mG}{\m{G}}
\newcommand{\mH}{\m{H}}
\newcommand{\mV}{\m{V}}
\newcommand{\mW}{\m{W}}
\newcommand{\mX}{\m{X}}
\newcommand{\mZ}{\m{Z}}
\newcommand{\mK}{\m{K}}
\newcommand{\mM}{\m{M}}

\newcommand{\meye}{\m{I}}

\newcommand{\ee}[1]{\times 10^{#1}}
\newcommand{\jac}[2]{\frac{\bfd \bm{#1}}{\bfd \bm{#2}}}
\newcommand{\diag}{\operatorname{diag}}
\newcommand{\fl}{\operatorname{fl}}
\newcommand{\circop}[1]{\makebox[0pt][l]{$\bigcirc$}\hspace{1pt}#1}
\newcommand{\myvec}{\operatorname{vec}}
\newcommand{\unvec}{\operatorname{unvec}}
\newcommand{\kron}[2]{#1 \otimes #2}

% matlab stuff
\lstset{style=Matlab-editor,basicstyle=\mlttfamily}
\newcommand{\Mfile}[1]{\lstinputlisting[]{#1}}
\lstnewenvironment{matlab}%
  {\lstset{escapechar=`}}%
  {}
\lstMakeShortInline[style=Matlab-editor,basicstyle=\mlttfamily]"  % use " for inline code
\newcommand{\matlabend}{\lstinline[style=Matlab-style,basicstyle=\mlttfamily,mloverride=true]!end!}


\begin{document}

\begin{center}
  \bf Rooting for accuracy
\end{center}

Suppose we have a quadratic polynomial $p(x;s) = a(s)x^2+b(s)x+c(s)$ whose coefficients depend on a parameter $s$. In this lab, we define
\begin{equation}
  \label{eq:poly}
  p(x;s) = 9x^2 - (6+s)x + 1.
\end{equation}
For each value of $s$ there are two roots $t_1(s)$ and $t_2(s)$. Since $a$, $b$, and $c$ vary continuously as a function of $s$, $t_1(s)$ and $t_2(s)$ are also continuous.

The effect of roundoff in floating point calculations is to perturb the coefficients of $p$ by a relative amount $\macheps$. As a result, the roots may change by a relative amount $\kappa \macheps$, where $\kappa$ is the condition number of the roots. In the text it's shown in a special case that for the root $t_k$, the condition number is given by
\begin{equation}
  \label{eq:kappa}
  \kappa = |t_k|/|t_1-t_2|.
\end{equation}
If $t_1\approx t_2$, the errors in the computed roots are potentially large (compared to machine precision). In our case $p(x;0)=(3x-1)^2$, which makes $\kappa$ formally infinite, so $s=0$ is critical to the conditioning of the roots.


%Even when a computed $\tilde{t}_k$ is not close to $t_k$, however, it may still be the case that the \textbf{residual} $p(\tilde{t}_k;s)$ is close to zero. In practical computations we usually don't know the error, but we can always compute the residual.

\subsection*{Goals}

You will examine the sensitivity of the roots of a quadratic polynomial whose roots come together, coalesce, and separate again.

\subsection*{Preparation}

Read section~1.2 and the online help for the \texttt{roots} and \texttt{polyval} commands.

In MATLAB, perform the following.
\begin{verbatim}
t1 = @(s) (6+s+sqrt((6+s).^2-36))/18;
t2 = @(s) (6+s-sqrt((6+s).^2-36))/18;
s = linspace(-1,1,800)';
plot(t1(s),'r.')
hold on
plot(t2(s),'b.')
xlabel('real part'),  ylabel('imaginary part')
title('Roots for -1 < s < 1') 
\end{verbatim}
Use the resulting picture to describe in words what happens to the roots as the parameter $s$ passes from negative to positive values. 


\subsection*{Procedure}

Download the template script and complete it in sections as guided below.

\begin{enumerate}
\item Define the functions \texttt{t1} and \texttt{t2} as in the Preparation section above.

\item Let $s=-0.1$. Use \texttt{roots} to compute the two roots of $p(x;s)$ as a vector \texttt{r}. One complication is that we don't know whether \texttt{r(1)} is meant to be $t_1$ or $t_2$. So we compute its error by means of
\begin{verbatim}
err1 = min( abs(r(1)-t1(s)), abs(r(1)-t2(s)) )
\end{verbatim}
  Find this value. Then find $\kappa$ from~\eqref{eq:kappa} with $k=1$.

\item Define the vector
\begin{verbatim}
s = linspace(-1,1,800)';
\end{verbatim}
  In a loop over the entries of the vector \texttt{s}, find the error and $\kappa$, storing the results in vectors. The template has code for plotting the results. 

\item Nothing in the previous step indicates much reason for concern. However, the results are strikingly different if we approach very close to $s=0$. Define 
\begin{verbatim}
s = logspace(-15,-1,100)';
\end{verbatim}
Then repeat the loop from step~3. 
\end{enumerate}

\subsection*{Going further}

We always compute with a fixed value of $\varepsilon_{\text{mach}}$, but we can simulate the effect of varying it by manually perturbing data. For example, a relative perturbation of size $\delta$ to each entry of vector $y$ is made by
\begin{verbatim}
r = 2*rand(size(y))-1;    % random in [-1,1]
yy = y .* (1+delta*r);
\end{verbatim}
For each $\delta=10^{-6},10^{-8},10^{-10},10^{-12},10^{-14}$, make 500 random perturbations to the coefficient vector of $p(x;0)$, each time using \texttt{roots} to find the resulting roots and plotting them in the complex plane. (You should end up five plots, each with 1000 points.) 


\end{document}

%%% Local Variables:
%%% mode: latex
%%% TeX-master: t
%%% End:
