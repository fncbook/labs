\documentclass[11pt,twoside]{article}

\usepackage{amsmath}
\usepackage[top=1in,bottom=0.75in,left=1in,right=1in]{geometry}
\usepackage[charter]{mathdesign}
\usepackage{color}

\usepackage{amsmath}
\usepackage{bm}
\usepackage[T1]{fontenc}
\usepackage{matlab-prettifier}

\newcommand{\nat}{\mathbb{N}}          % Natural numbers
\newcommand{\integer}{\mathbb{Z}}      % Integers
\newcommand{\real}{ {\mathbb{R}} }     % Reals
\newcommand{\float}{ {\mathbb{F}} }     % Reals
\newcommand{\rmn}[2]{ \mathbb{R}^{#1\times#2} }     % Reals
\newcommand{\complex}{ {\mathbb{C}} }  % Complex
\newcommand{\macheps}{\ensuremath \varepsilon_{\text{mach}}}

\renewcommand{\Re}{\operatorname{Re}}
\renewcommand{\Im}{\operatorname{Im}}

% Boldface vectors
\newcommand{\bff}{\bm{f}}
\newcommand{\bfF}{\bm{F}}
\newcommand{\bfw}{\bm{w}}
\newcommand{\bfv}{\bm{v}}
\newcommand{\bfe}{\bm{e}}
\newcommand{\bfc}{\bm{c}}
\newcommand{\bfp}{\bm{p}}
\newcommand{\bfq}{\bm{q}}
\newcommand{\bfr}{\bm{r}}
\newcommand{\bfs}{\bm{s}}
\newcommand{\bfu}{\bm{u}}
\newcommand{\bfb}{\bm{b}}
\newcommand{\bfx}{\bm{x}}
\newcommand{\bfy}{\bm{y}}
\newcommand{\bfg}{\bm{g}}
\newcommand{\bfh}{\bm{h}}
\newcommand{\bfz}{\bm{z}}
\newcommand{\bfa}{\bm{a}}
\newcommand{\bft}{\bm{t}}
\newcommand{\bfd}{\bm{d}}
\newcommand{\bfalpha}{\bm{\alpha}}
\newcommand{\bfeps}{\bm{\varepsilon}}
\newcommand{\bfdelta}{\bm{\delta}}
\newcommand{\bfzero}{\bm{0}}
\newcommand{\eye}[1]{\bfe_{#1}}

% Boldface matrix
\newcommand{\m}[1]{\bm{#1}}
\newcommand{\mA}{\m{A}}
\newcommand{\mL}{\m{L}}
\newcommand{\mF}{\m{F}}
\newcommand{\mU}{\m{U}}
\newcommand{\mJ}{\m{J}}
\newcommand{\mP}{\m{P}}
\newcommand{\mQ}{\m{Q}}
\newcommand{\mR}{\m{R}}
\newcommand{\mD}{\m{D}}
\newcommand{\mS}{\m{S}}
\newcommand{\mB}{\m{B}}
\newcommand{\mC}{\m{C}}
\newcommand{\mE}{\m{E}}
\newcommand{\mG}{\m{G}}
\newcommand{\mH}{\m{H}}
\newcommand{\mV}{\m{V}}
\newcommand{\mW}{\m{W}}
\newcommand{\mX}{\m{X}}
\newcommand{\mZ}{\m{Z}}
\newcommand{\mK}{\m{K}}
\newcommand{\mM}{\m{M}}

\newcommand{\meye}{\m{I}}

\newcommand{\ee}[1]{\times 10^{#1}}
\newcommand{\jac}[2]{\frac{\bfd \bm{#1}}{\bfd \bm{#2}}}
\newcommand{\diag}{\operatorname{diag}}
\newcommand{\fl}{\operatorname{fl}}
\newcommand{\circop}[1]{\makebox[0pt][l]{$\bigcirc$}\hspace{1pt}#1}
\newcommand{\myvec}{\operatorname{vec}}
\newcommand{\unvec}{\operatorname{unvec}}
\newcommand{\kron}[2]{#1 \otimes #2}

% matlab stuff
\lstset{style=Matlab-editor,basicstyle=\mlttfamily}
\newcommand{\Mfile}[1]{\lstinputlisting[]{#1}}
\lstnewenvironment{matlab}%
  {\lstset{escapechar=`}}%
  {}
\lstMakeShortInline[style=Matlab-editor,basicstyle=\mlttfamily]"  % use " for inline code
\newcommand{\matlabend}{\lstinline[style=Matlab-style,basicstyle=\mlttfamily,mloverride=true]!end!}
 

\begin{document}

\begin{center}
  \Large
  \bf What goes around, comes around
\end{center}

\section{Plane truth}

A small satellite in the Earth--moon system can be modeled as a \emph{restricted three-body problem}, in which one object has so little relative mass that the motions essentially take place in a plane. In this formulation, Newton's laws of motion are simplified by some transformations and conveniently chosen units of measure. 

The value $\mu<0.5$ is the ratio of the masses of the bodies, and the coordinates have origin at the center of mass and rotate so that Earth remains at $(-\mu,0)$, while the moon stays at $(1-\mu,0)$. For the Earth/moon system, $\mu=0.012277471$.

The position of the satellite is $(x(t),y(t))$ and satisfies
\begin{equation}
  \label{eq:r3body}
\begin{split}
  x'' &= x + 2y' - (1-\mu) \frac{x+\mu}{r_e^3} - \mu \frac{x-1+\mu}{r_m^3} \\
  y'' &= y - 2x' - (1-\mu) \frac{y}{r_e^3} - \mu \frac{y}{r_m^3} 
\end{split}
\end{equation}
where
\begin{equation}
  \label{eq:rrstar}
  \begin{split}
  r_e &= \left[ (x+\mu)^2 + y^2 \right]^{1/2} \\
  r_m &= \left[ (x-1+\mu)^2 + y^2 \right]^{1/2}
  \end{split}
\end{equation}
are the distances of the satellite to Earth and moon, respectively. The key to computing solutions using standard software is to transform the original ODE system~\eqref{eq:r3body} into a first-order one. 

Define $u_1=x$, $u_2=y$, $u_3=x'$, $u_4=y'$. Two equations in the new system are thus 
\begin{align}
  \frac{du_1}{dt} &= u_3  \label{eq:tb1} \\
  \frac{du_2}{dt} &= u_4  \label{eq:tb2}
\end{align}
The other two equations come from substitution into the original system. Since $\frac{du_3}{dt}=\frac{d^2x}{dt^2}$, we get
\begin{equation}
  \label{eq:tb3}
  \frac{du_3}{dt} = u_1 + 2u_4 - (1-\mu) \frac{u_1+\mu}{r_e^3} - \mu \frac{u_1-1+\mu}{r_m^3}
\end{equation}
Similarly, from $\frac{du_4}{dt}=\frac{d^2y}{dt^2}$ we get
\begin{equation}
  \label{eq:tb4}
  \frac{du_4}{dt} = u_2 - 2u_3 - (1-\mu) \frac{u_2}{r_e^3} - \mu \frac{u_2}{r_m^3}
\end{equation}
Altogether we now have a vector equation $\bm{u}'(t) = \bm{f}(t,\bm{u}(t))$, where $u_i'=f_i$ as given in the four equations above. 

\subsection*{Procedure}
\begin{enumerate}
  \item \textbf{AT THE END OF YOUR FILE}, define a function 
  \begin{matlab}
  function f = r3body(t,u)
  \end{matlab}
  This function should compute a $4\times 1$ vector "f" whose components are given as in equations~\eqref{eq:tb1}--\eqref{eq:tb4}. 
  \item To check your function, call it with "t=0" and "u=[1;2;3;4]" (back in the main part of your script). The result should be about "[3;4;8.911;-4.178]".
\end{enumerate}

\section{(Not) chaos}

The full three-body problem famously exhibits chaotic behavior. In our case, with the satellite so small and the motion restricted to a plane, chaos isn't possible. But the motions can be very complex-looking.

\subsection*{Procedure}
\begin{enumerate}
  \item Set
  \begin{verbatim}
  opt = odeset('reltol',1e-13,'abstol',1e-13);
  \end{verbatim}
  This will be used \textbf{in all of your ODE solutions below} to require strict error tolerances. It should be passed to "ode113" as a fourth input argument. 
  \item Consider the initial conditions 
  \begin{align*}
    x(0) &= \alpha, \; & x'(0) &= 0, \\
    y(0) &= 0, \; & y'(0) &= 0.
  \end{align*}
  Using "ode113" with "opt" as fourth input argument, solve the system with $\alpha = 0.3$ at 2000 time points in the interval $[0,10]$. Plot $x$ and $y$ as functions of $t$ on one graph. (Use a legend and label the time axis.) It will look like a couple of very ``bumpy'' sine waves.
  \item Using the same solution, clear graphics and make a phase plane plot ($x$ on the horizontal axis, $y$ on the vertical axis). Add the following to put both bodies on the graph and set a 1:1 aspect ratio:
  \begin{matlab}
  hold on
  plot(-mu,0,'go'), plot(1-mu,0,'kx')
  axis equal
  \end{matlab}
  Label the axes.
  \item Clear graphics and make a phase plot for $\alpha=0.7$.
  \item Clear graphics and make a phase plot for $\alpha=-0.2$.
\end{enumerate}


\section{Figure eight}

Before the first moon landing, Richard Arenstorf at NASA discovered an orbit with just two loops that looks like an ice skater's figure eight. He thought this would be ideal for a ``space bus'' between Earth and moon some day, because it is more energy-efficient than a more direct route. 

\subsection*{Procedure}

\begin{enumerate}
 \item Using the initial conditions 
  \begin{align*}
    x(0) &= 1.2, \; & x'(0) &= 0, \\
    y(0) &= 0, \; & y'(0) &= -1.049357510,
  \end{align*}
  solve the problem at 2000 time points on the interval $[0,10]$, using "ode113" and passing "opt" as a fourth input argument. Clear graphics and make a phase plot. You should see what looks like an infinity symbol.
  \item The phase plane masks some of the dynamics. Using the computed solution, define a vector "speed" that is the speed of the satellite over time. (Recall that speed is the square root of $(x')^2+(y')^2$.) Make sure that "x" and "y" are vectors describing the position. Then this code uses color to plot the speed around the orbit:
  \begin{matlab}
  scatter(x,y,4,speed)
  axis equal, colorbar
  \end{matlab}
  As you would intuitively expect, the satellite moves fastest as it zooms by Earth. 
\end{enumerate}


\section{Loop-di-loop}

There is a whole family of orbits like Arenstorf's, with more and more loops in each complete orbit. These orbits are very sensitive to initial conditions and errors, however, and numerical solutions may depart from the true orbits rather dramatically as $t\to\infty$. (Remember that convergence is guaranteed over a fixed interval as the step size goes to zero.)


\subsection*{Procedure}

\begin{enumerate}
  \item Solve for
  \begin{align*}
    x(0) &= 0.994, \; & x'(0) &= 0, \\
    y(0) &= 0, \; & y'(0) &= -2.03173262955734,
  \end{align*}
  at 2000 points over $[0,20]$. As in the previous section, make a phase plot with coloring to show the speed of the motion. This orbit has three loops.
  \item Repeat the previous step for  
      \begin{align*}
        x(0) &= 0.994, \; & x'(0) &= 0, \\
        y(0) &= 0, \; & y'(0) &= -2.00158510637908.
      \end{align*}
      This one has four loops, and now the fastest speed is during the fly-by of the moon.
  \item Repeat the previous step for 5000 time points over $[0,100]$. You should see the orbit deviate and then the satellite veer off out of the system entirely.
\end{enumerate}


\end{document}

%%% Local Variables: 
%%% mode: latex
%%% TeX-master: t
%%% End: 
