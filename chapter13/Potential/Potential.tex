\documentclass[11pt,twoside]{article}

\usepackage[headings]{fullpage}
\usepackage[utopia]{mathdesign}

\pagestyle{myheadings}
\markboth{Potential}{Potential}

\usepackage{amsmath}
\usepackage{bm}
\usepackage[T1]{fontenc}
\usepackage{matlab-prettifier}

\newcommand{\nat}{\mathbb{N}}          % Natural numbers
\newcommand{\integer}{\mathbb{Z}}      % Integers
\newcommand{\real}{ {\mathbb{R}} }     % Reals
\newcommand{\float}{ {\mathbb{F}} }     % Reals
\newcommand{\rmn}[2]{ \mathbb{R}^{#1\times#2} }     % Reals
\newcommand{\complex}{ {\mathbb{C}} }  % Complex
\newcommand{\macheps}{\ensuremath \varepsilon_{\text{mach}}}

\renewcommand{\Re}{\operatorname{Re}}
\renewcommand{\Im}{\operatorname{Im}}

% Boldface vectors
\newcommand{\bff}{\bm{f}}
\newcommand{\bfF}{\bm{F}}
\newcommand{\bfw}{\bm{w}}
\newcommand{\bfv}{\bm{v}}
\newcommand{\bfe}{\bm{e}}
\newcommand{\bfc}{\bm{c}}
\newcommand{\bfp}{\bm{p}}
\newcommand{\bfq}{\bm{q}}
\newcommand{\bfr}{\bm{r}}
\newcommand{\bfs}{\bm{s}}
\newcommand{\bfu}{\bm{u}}
\newcommand{\bfb}{\bm{b}}
\newcommand{\bfx}{\bm{x}}
\newcommand{\bfy}{\bm{y}}
\newcommand{\bfg}{\bm{g}}
\newcommand{\bfh}{\bm{h}}
\newcommand{\bfz}{\bm{z}}
\newcommand{\bfa}{\bm{a}}
\newcommand{\bft}{\bm{t}}
\newcommand{\bfd}{\bm{d}}
\newcommand{\bfalpha}{\bm{\alpha}}
\newcommand{\bfeps}{\bm{\varepsilon}}
\newcommand{\bfdelta}{\bm{\delta}}
\newcommand{\bfzero}{\bm{0}}
\newcommand{\eye}[1]{\bfe_{#1}}

% Boldface matrix
\newcommand{\m}[1]{\bm{#1}}
\newcommand{\mA}{\m{A}}
\newcommand{\mL}{\m{L}}
\newcommand{\mF}{\m{F}}
\newcommand{\mU}{\m{U}}
\newcommand{\mJ}{\m{J}}
\newcommand{\mP}{\m{P}}
\newcommand{\mQ}{\m{Q}}
\newcommand{\mR}{\m{R}}
\newcommand{\mD}{\m{D}}
\newcommand{\mS}{\m{S}}
\newcommand{\mB}{\m{B}}
\newcommand{\mC}{\m{C}}
\newcommand{\mE}{\m{E}}
\newcommand{\mG}{\m{G}}
\newcommand{\mH}{\m{H}}
\newcommand{\mV}{\m{V}}
\newcommand{\mW}{\m{W}}
\newcommand{\mX}{\m{X}}
\newcommand{\mZ}{\m{Z}}
\newcommand{\mK}{\m{K}}
\newcommand{\mM}{\m{M}}

\newcommand{\meye}{\m{I}}

\newcommand{\ee}[1]{\times 10^{#1}}
\newcommand{\jac}[2]{\frac{\bfd \bm{#1}}{\bfd \bm{#2}}}
\newcommand{\diag}{\operatorname{diag}}
\newcommand{\fl}{\operatorname{fl}}
\newcommand{\circop}[1]{\makebox[0pt][l]{$\bigcirc$}\hspace{1pt}#1}
\newcommand{\myvec}{\operatorname{vec}}
\newcommand{\unvec}{\operatorname{unvec}}
\newcommand{\kron}[2]{#1 \otimes #2}

% matlab stuff
\lstset{style=Matlab-editor,basicstyle=\mlttfamily}
\newcommand{\Mfile}[1]{\lstinputlisting[]{#1}}
\lstnewenvironment{matlab}%
  {\lstset{escapechar=`}}%
  {}
\lstMakeShortInline[style=Matlab-editor,basicstyle=\mlttfamily]"  % use " for inline code
\newcommand{\matlabend}{\lstinline[style=Matlab-style,basicstyle=\mlttfamily,mloverride=true]!end!}

\newcommand{\pp}[2]{\frac{\partial #1}{\partial #2}}

\begin{document}

\begin{center}
  \bf You've got potential
\end{center}

One reason the Laplace equation is considered so important is its relationship to \emph{potential functions}. Broadly speaking, a potential function usually replaces a vector quantity with a scalar one. The standard relationship is via a gradient, e.g., $\bfF=\nabla u$. Familiar forces from physics are subject to this relationship, as are certain fluid flows and other phenomena. 

In the absence of sources or sinks, a conservative force is divergence-free, which implies that $\nabla^2 u=0$. So Laplace's equation can be solved for the potential, which can be considerably easier than solving the equations for the orignal force. Level curves of $u(x,y)$ are called \emph{equipotential curves}. Thanks to the properties of the gradient, the force is everywhere orthogonal to the equipotential curves. 

\subsection*{Goals}

You will use numerical solutions of Laplace's equation to find the potential and force for a two different boundary conditions on a square.

\subsection*{Preparation}

Read section 13.3. All of the following questions are about the potential function $u(x,y)=(\cos \theta)x + (\sin \theta)y$, for an arbitrary fixed real $\theta$.
\begin{enumerate}
\item Show that $u$ is a solution of Laplace's equation.
\item Find $\nabla u$.
\item Suppose $\theta=\pi/3$. In the square $[0,1]\times[0,1]$, sketch 5 equipotential curves. Draw arrows representing the gradient at 3 points along each equipotential curve. 
\end{enumerate}

\subsection*{Procedure}

Download the template script and complete it to perform the following tasks.

\begin{enumerate}
\item Let the boundary values be given by $g(x,y) = \tanh\bigl(12 w(x,y) - 7\bigr)$, where $w(x,y)=(\cos \theta)x + (\sin \theta)y$ and $\theta=\pi/5$. Solve Laplace's equation on $[0,1]\times[0,1]$ with $u=g$ on the boundary and $m=n=144$. Make a surface plot of the solution, labelling the axes.
  
\item Use differentiation matrices to compute $\pp{u}{x}$ and $\pp{u}{y}$ on the grid. Use a $1\times2$ subplot grid to make surface plots of them.
  
\item In a new graph, use \texttt{contour} to make a contour plot of $u$ with 32 level curves. To this plot you should add a \texttt{quiver} plot that draws arrows representing the gradient of $u$. In order to avoid having too many arrows, use only every 6th row and column in all of the matrix arguments to \texttt{quiver}. Use \texttt{axis equal} after the plot.
  
\item Repeat the previous steps for $\theta=3\pi/5$.
\end{enumerate}

\subsection*{Extra}
\begin{enumerate}
\item[E1.] In two dimensions, solutions of Laplace's equation have a close connection with complex analysis. For instance, let $z=x+iy$, let $\zeta=h(z)$ for a differentiable function $h$ whose derivative never vanishes in the square, and let $\xi=\Re(\zeta)$, $\eta=\Im(\zeta)$. If $u_{xx}+u_{yy}=0$, and if we define
\[
\tilde{u}(\xi,\eta) = u(x,y),
\]
then $\tilde{u}_{\xi\xi} + \tilde{u}_{\eta\eta}=0$ and $\tilde{u}$ is a potential function in the geometry that results from applying $h$ to every point in the $(x,y)$ domain.

Let $h(z)=\exp(\pi z/2)$. Calculate $u$ as in step~1 above. Then make a \texttt{pcolor} plot of $\tilde{u}$ over its domain (which should look like a quarter of a donut). Don't forget \texttt{axis equal}!
\end{enumerate}



\end{document}

%%% Local Variables: 
%%% mode: latex
%%% TeX-master: t
%%% End: 
