\documentclass[11pt,twoside]{article}

\pagestyle{myheadings}
\markboth{Actor ranking}{Actor ranking}

\usepackage{amsmath}
\usepackage{bm}
\usepackage[T1]{fontenc}
\usepackage{matlab-prettifier}

\newcommand{\nat}{\mathbb{N}}          % Natural numbers
\newcommand{\integer}{\mathbb{Z}}      % Integers
\newcommand{\real}{ {\mathbb{R}} }     % Reals
\newcommand{\float}{ {\mathbb{F}} }     % Reals
\newcommand{\rmn}[2]{ \mathbb{R}^{#1\times#2} }     % Reals
\newcommand{\complex}{ {\mathbb{C}} }  % Complex
\newcommand{\macheps}{\ensuremath \varepsilon_{\text{mach}}}

\renewcommand{\Re}{\operatorname{Re}}
\renewcommand{\Im}{\operatorname{Im}}

% Boldface vectors
\newcommand{\bff}{\bm{f}}
\newcommand{\bfF}{\bm{F}}
\newcommand{\bfw}{\bm{w}}
\newcommand{\bfv}{\bm{v}}
\newcommand{\bfe}{\bm{e}}
\newcommand{\bfc}{\bm{c}}
\newcommand{\bfp}{\bm{p}}
\newcommand{\bfq}{\bm{q}}
\newcommand{\bfr}{\bm{r}}
\newcommand{\bfs}{\bm{s}}
\newcommand{\bfu}{\bm{u}}
\newcommand{\bfb}{\bm{b}}
\newcommand{\bfx}{\bm{x}}
\newcommand{\bfy}{\bm{y}}
\newcommand{\bfg}{\bm{g}}
\newcommand{\bfh}{\bm{h}}
\newcommand{\bfz}{\bm{z}}
\newcommand{\bfa}{\bm{a}}
\newcommand{\bft}{\bm{t}}
\newcommand{\bfd}{\bm{d}}
\newcommand{\bfalpha}{\bm{\alpha}}
\newcommand{\bfeps}{\bm{\varepsilon}}
\newcommand{\bfdelta}{\bm{\delta}}
\newcommand{\bfzero}{\bm{0}}
\newcommand{\eye}[1]{\bfe_{#1}}

% Boldface matrix
\newcommand{\m}[1]{\bm{#1}}
\newcommand{\mA}{\m{A}}
\newcommand{\mL}{\m{L}}
\newcommand{\mF}{\m{F}}
\newcommand{\mU}{\m{U}}
\newcommand{\mJ}{\m{J}}
\newcommand{\mP}{\m{P}}
\newcommand{\mQ}{\m{Q}}
\newcommand{\mR}{\m{R}}
\newcommand{\mD}{\m{D}}
\newcommand{\mS}{\m{S}}
\newcommand{\mB}{\m{B}}
\newcommand{\mC}{\m{C}}
\newcommand{\mE}{\m{E}}
\newcommand{\mG}{\m{G}}
\newcommand{\mH}{\m{H}}
\newcommand{\mV}{\m{V}}
\newcommand{\mW}{\m{W}}
\newcommand{\mX}{\m{X}}
\newcommand{\mZ}{\m{Z}}
\newcommand{\mK}{\m{K}}
\newcommand{\mM}{\m{M}}

\newcommand{\meye}{\m{I}}

\newcommand{\ee}[1]{\times 10^{#1}}
\newcommand{\jac}[2]{\frac{\bfd \bm{#1}}{\bfd \bm{#2}}}
\newcommand{\diag}{\operatorname{diag}}
\newcommand{\fl}{\operatorname{fl}}
\newcommand{\circop}[1]{\makebox[0pt][l]{$\bigcirc$}\hspace{1pt}#1}
\newcommand{\myvec}{\operatorname{vec}}
\newcommand{\unvec}{\operatorname{unvec}}
\newcommand{\kron}[2]{#1 \otimes #2}

% matlab stuff
\lstset{style=Matlab-editor,basicstyle=\mlttfamily}
\newcommand{\Mfile}[1]{\lstinputlisting[]{#1}}
\lstnewenvironment{matlab}%
  {\lstset{escapechar=`}}%
  {}
\lstMakeShortInline[style=Matlab-editor,basicstyle=\mlttfamily]"  % use " for inline code
\newcommand{\matlabend}{\lstinline[style=Matlab-style,basicstyle=\mlttfamily,mloverride=true]!end!}


\begin{document}
    
\begin{center}
  \bf Not Kevin Bacon
\end{center}
    
Suppose we have a graph or network with $n$ nodes and some connections between them. The matrix $\mA$ whose entries are 
\begin{equation}
  A_{ij} = 
  \begin{cases}
    1, & \text{if node $i$ connects to $j$},\\
    0, & \text{otherwise,}
  \end{cases}
\end{equation}
is the \textbf{adjacency matrix} of the network. Note that these connections have direction, and $i$ connecting to $j$ does not automatically imply that $j$ connects to $i$ (i.e., this is a directed graph). Let $s_i = \sum_{j=1}^n A_{ij}$ be the number of nodes that node $i$ connects to. If we were to randomly select a link leaving node $i$, each link would have probability $1/s_i$ of being selected. 
     
Let $\bfx$ be a vector of positive values. We will require that $\sum_i x_i = 1$, so that $\bfx$ has the interpretation of a probability distribution over the nodes. If all connections are given equal weighting, the probability of following a connection from any node to node $i$ is
\begin{equation}
  z_i = \sum_{j\in P_i} \frac{x_j}{s_j} ,
\end{equation} 
where $P_i$ is the set of nodes that connect to node $i$. These are the rows with ones in column $i$ of $\mA$. By the definition of $\mA$, this is the same as  
\begin{equation}
  \label{eq:yvec}
  z_i = \sum_{j=1}^n \frac{A_{ji} x_j}{s_j} =\sum_{j=1}^n B_{ij} x_j,
\end{equation} 
where we defined $B_{ij}=A_{ji}/s_j$. Put simply, $\bfz=\mB \bfx$. 

It's important to introduce some overall randomness into the jumps between nodes---this is the only way to escape a self-contained clique (disconnected subgraph). The probability of hopping to any node $i$ entirely at random is just $1/n$. We blend link-following with random hopping as follows. Choose some $p\in[0,1]$, and suppose that a hop between nodes follows one of the connections with probability $p$, or is a random hop with probability $1-p$. Using $\bm{1}$ to denote the $n$-vector of all ones, then 
\begin{equation}
  \label{eq:totalmap}
  \bfy = p \mB \bfx + \frac{1-p}{n} \bm{1}
\end{equation}
describes how to update probabilities after each hop. Finally, the fact that 
\begin{equation}
  1 = \sum_i x_i = \bm{1}^T \bfx,
\end{equation}
allows us to express the map~\eqref{eq:totalmap} as
\begin{equation}
  \label{eq:matvec}
  \bfy =  \left[ p \mB + \frac{1-p}{n} \bm{1}\bm{1}^T \right] \bfx = \mR\bfx,
\end{equation}  
for a square matrix $\mR$. If the probabilities are unchanged by hopping (i.e., $\bfz=\bfx$), then $\bfx$ is an eigenvector of $\mR$ with associated eigenvalue $\lambda=1$. We won't prove this, but $\mR$ is guaranteed to have $\lambda=1$ as the leading eigenvalue, making power iteration possible. The resulting eigenvector $\bfx$ can be sorted to find out which nodes are most likely to be visited in the long run.

Note that $\mR$ is \emph{not} sparse and should never be formed. However, $\mR\bfx$ as defined in~\eqref{eq:totalmap} can be computed efficiently if $\mB$ is sparse. That is all we need to do a power iteration with $\mR$. Since we are working with probability, normalization is not required in the power iteration: $\bfx_{k+1}=\mR \bfx_k$, provided $\bfx_1$ has positive entries and $\|\bfx\|_1=1$.  


\subsection*{Goals}

You will use an adjacency matrix for movie actors to perform the power iteration and find the leading eigenvector of $\mR$, and using that vector to rank the actors in influence.
    

\subsection*{Preparation}

Read section 8.2. Answer the following questions based on the above description.

\begin{enumerate}
\item Show using \eqref{eq:yvec} that $\displaystyle \sum_{i=1}^n z_i = 1$. This proves that $\bfz$ is also a probability distribution.
  \item Show using \eqref{eq:totalmap} that $\bfy$ is a probability distribution. 
\end{enumerate}
 
%\vspace{1ex}     
%\noindent \textbf{Notable MATLAB functions:} \texttt{whos}, \texttt{nnz}, \texttt{sort}, \texttt{histogram}

    
\subsection*{Procedure}

Download the script template and the data file \texttt{actornetwork.mat}.

\begin{enumerate}
\item Load \texttt{actornetwork.mat} file from the assignment site. It has a vector \texttt{actor} of unique actor names for all credited roles in films released from 2004 through 2013. It also has a sparse adjacency matrix \texttt{A}, where a (symmetric) link between actors means that they appeared in at least one film together.
\item Use \texttt{nnz} to compute the density (number of nonzeros over total number of elements) of \texttt{A}. Use \texttt{whos} to find the memory usage of \texttt{A} in bytes. Also calculate the memory usage of an equivalent full (non-sparse) matrix. 
\item Compute the vector $\mathbf{s}$ whose entries are $s_i$ as defined above. Make a histogram of its entries using 32 bins.  
\item  Construct the matrix $\mB$ appearing in~\eqref{eq:yvec} above. It helps to use the fact that $\mA$ is symmetric. (It's reasonable to loop over one dimension of the matrix, but not over all of the elements.)
\item Set $p=0.9$ and let $\bfx_1$ be a random vector of positive numbers. Normalize $\bfx_1$ to be a probability distribution. By repeatedly applying~\eqref{eq:totalmap}, do 100 power iterations to get $\bfx_{101}$. \emph{Important: Do not attempt to define the matrix $\mR$}, and do \emph{not} use the book's power iteration function; instead use~\eqref{totalmap} to compute $\mR\bfx$. Check that $\|\bfx_{101}-\bfx_{100}\|_1$ is less than $10^{-6}$. 
\item Sort the entries of $\bfx$ in descending order, using the second output to print out the names of the 10 ``most collaborative'' actors. 
\end{enumerate}

\subsection*{Discussion}
\begin{enumerate}
\item The data file also includes an $m\times n$ sparse matrix $\m{M}$. Its $(i,j)$ entry is one if actor $j$ appeared in film $i$, and zero otherwise. Give a simple interpretation of the matrix $\m{M}^T\m{M}$.
%\item[E2.] Repeat the ranking process using $\m{M}^T\m{M}$ in place of $\mA$. How many movies is the top actor credited with appearing in? 
\item What is the interpretation of $\m{M}\m{M}^T$? What would a ranking using this matrix reveal? 
\end{enumerate}
    
    
    
\end{document}

